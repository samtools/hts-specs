\section{The VCF specification}
VCF is a text file format (most likely stored in a compressed manner). 
It contains meta-information lines (prefixed with "\#\#"), a header
line (prefixed with "\#"), and data lines
each containing information about a position in the genome and genotype
information on samples for each position
(text fields separated by tabs). Zero length fields are not allowed, a dot (".") should
be used instead.
In order to ensure interoperability across platforms, VCF compliant implementations must support
both LF (\texttt{\textbackslash n}) and CR+LF (\texttt{\textbackslash r\textbackslash n}) newline conventions.  

\subsection{An example}
\scriptsize
\begin{verbatim}
##fileformat=VCFv4.3
##fileDate=20090805
##source=myImputationProgramV3.1
##reference=file:///seq/references/1000GenomesPilot-NCBI36.fasta
##contig=<ID=20,length=62435964,assembly=B36,md5=f126cdf8a6e0c7f379d618ff66beb2da,species="Homo sapiens",taxonomy=x>
##phasing=partial
##INFO=<ID=NS,Number=1,Type=Integer,Description="Number of Samples With Data">
##INFO=<ID=DP,Number=1,Type=Integer,Description="Total Depth">
##INFO=<ID=AF,Number=A,Type=Float,Description="Allele Frequency">
##INFO=<ID=AA,Number=1,Type=String,Description="Ancestral Allele">
##INFO=<ID=DB,Number=0,Type=Flag,Description="dbSNP membership, build 129">
##INFO=<ID=H2,Number=0,Type=Flag,Description="HapMap2 membership">
##FILTER=<ID=q10,Description="Quality below 10">
##FILTER=<ID=s50,Description="Less than 50% of samples have data">
##FORMAT=<ID=GT,Number=1,Type=String,Description="Genotype">
##FORMAT=<ID=GQ,Number=1,Type=Integer,Description="Genotype Quality">
##FORMAT=<ID=DP,Number=1,Type=Integer,Description="Read Depth">
##FORMAT=<ID=HQ,Number=2,Type=Integer,Description="Haplotype Quality">
#CHROM POS     ID        REF    ALT     QUAL FILTER INFO                              FORMAT      NA00001        NA00002        NA00003
20     14370   rs6054257 G      A       29   PASS   NS=3;DP=14;AF=0.5;DB;H2           GT:GQ:DP:HQ 0|0:48:1:51,51 1|0:48:8:51,51 1/1:43:5:.,.
20     17330   .         T      A       3    q10    NS=3;DP=11;AF=0.017               GT:GQ:DP:HQ 0|0:49:3:58,50 0|1:3:5:65,3   0/0:41:3
20     1110696 rs6040355 A      G,T     67   PASS   NS=2;DP=10;AF=0.333,0.667;AA=T;DB GT:GQ:DP:HQ 1|2:21:6:23,27 2|1:2:0:18,2   2/2:35:4
20     1230237 .         T      .       47   PASS   NS=3;DP=13;AA=T                   GT:GQ:DP:HQ 0|0:54:7:56,60 0|0:48:4:51,51 0/0:61:2
20     1234567 microsat1 GTC    G,GTCT  50   PASS   NS=3;DP=9;AA=G                    GT:GQ:DP    0/1:35:4       0/2:17:2       1/1:40:3
\end{verbatim}
\normalsize
This example shows (in order): a good simple SNP, a possible SNP that has been filtered out because its quality is below 10, a site at which two alternate alleles are called, with one of them (T) being ancestral (possibly a reference sequencing error), a site that is called monomorphic reference (i.e. with no alternate alleles), and a microsatellite with two alternative alleles, one a deletion of 2 bases (TC), and the other an insertion of one base (T). Genotype data are given for three samples, two of which are phased and the third unphased, with per sample genotype quality, depth and haplotype qualities (the latter only for the phased samples) given as well as the genotypes. The microsatellite calls are unphased.

\subsection{Character encoding, non-printable characters and characters with special meaning}
\label{character-encoding}
The character encoding of VCF files is UTF-8.  UTF-8 is a multi-byte 
character encoding that is a strict superset of 7-bit ASCII and has the 
property that none of the bytes in any multi-byte characters are 7-bit ASCII 
bytes. As a result, most software that processes VCF files does not have 
to be aware of the possible presence of multi-byte UTF-8 characters. 
Note that non-printable characters U+0000-U+0008, U+000B-U+000C, U+000E-U+001F are disallowed.
Line separators must be CR+LF or LF and they are allowed only as line separators at
end of line.
Characters with special meaning (such as field delimiters ';' in INFO 
or ':' FORMAT fields) must be represented using the capitalized percent encoding:

\begingroup\footnotesize
\begin{tabular}{l l l}
\%3A  &  :  & (colon)                \\
\%3B  &  ;  & (semicolon)            \\
\%3D  &  =  & (equal sign)           \\
\%25  &  \% & (percent sign)         \\
\%2C  &  ,  & (comma)                \\
\%0D  & CR  &                        \\
\%0A  & LF  &                        \\
\%09  & TAB & 
\end{tabular}
\endgroup


\subsection{Data types}
Data types supported by VCF are: Integer (32-bit, signed), Float (32-bit, formatted 
to match the regular expression \texttt{\^{}[-+]?[0-9]*\textbackslash.?[0-9]+([eE][-+]?[0-9]+)?\$}, \texttt{NaN}, or \texttt{+/-Inf}), Flag, Character, and
String. For the Integer type, the values from $-2^{31}$ to $-2^{31}+7$ cannot be stored in the binary version and therefore 
are disallowed in both VCF and BCF, see \ref{BcfTypeEncoding}.

\subsection{Meta-information lines}


File meta-information is included after the \#\# string and must be key=value
pairs. Meta-information lines are optional, but if they are present then
they must be completely well-formed. Note that BCF, the binary
counterpart of VCF, requires that all entries are present.  It is strongly
encouraged to include meta-information lines describing the entries used in the
body of the VCF file.

All structured lines that have their value enclosed within "$<>$" require an ID
which must be unique within their type. For all of the structured lines (\#\#INFO, \#\#FORMAT,
\#\#FILTER, etc.), extra fields can be included after the default fields. For example:
\begin{verbatim}
##INFO=<ID=ID,Number=number,Type=type,Description="description",Source="description",Version="128">
\end{verbatim}
In the above example, the extra fields of ``Source'' and ``Version'' are
provided. Optional fields should be stored as strings even for numeric values.

It is highly recommended (but not required) that the header
include tags describing the reference and contigs backing the data contained in
the file.  These tags are based on the SQ field from the SAM spec; all tags are
optional (see the VCF example above).

Meta-information lines can be in any order with the exception of `fileformat`
which must come first.


\subsubsection{File format}
A single `fileformat' line is always required, must be the first line in the file, and details the VCF format version number. For VCF version 4.3, this line should read:

\begin{verbatim}
##fileformat=VCFv4.3
\end{verbatim}



\subsubsection{Information field format}
INFO fields should be described as follows (first four keys are required, source and version are recommended):

\begin{verbatim}
##INFO=<ID=ID,Number=number,Type=type,Description="description",Source="source",Version="version">
\end{verbatim}

Possible Types for INFO fields are: Integer, Float, Flag, Character, and
String. 
The Number entry is an Integer that describes the number of values that
can be included with the INFO field. For example, if the INFO field contains a
single number, then this value should be $1$; if the INFO field describes a
pair of numbers, then this value should be $2$ and so on. There are also
certain special characters used to define special cases:

\begin{itemize}
  \item If the field has one value per alternate allele then this value should be `A'.
  \item If the field has one value for each possible allele (including the reference), then this value should be `R'.
  \item If the field has one value for each possible genotype (more relevant to the FORMAT tags) then this value should be `G'.
  \item If the number of possible values varies, is unknown, or is unbounded, then this value should be `.'.
\end{itemize}

The `Flag' type indicates that the INFO field does not contain a Value entry, and hence the Number should be $0$ in this case. The Description value must be surrounded by double-quotes. Double-quote character can be escaped with backslash $\backslash$ and backslash as $\backslash\backslash$. Source and Version values likewise should be surrounded by double-quotes and specify the annotation source (case-insensitive, e.g. ``dbsnp'') and exact version (e.g. ``138''), respectively for computational use.

\subsubsection{Filter field format}
FILTERs that have been applied to the data should be described as follows:

\begin{verbatim}
##FILTER=<ID=ID,Description="description">
\end{verbatim}

\subsubsection{Individual format field format}
Likewise, Genotype fields specified in the FORMAT field should be described as follows:

\begin{verbatim}
##FORMAT=<ID=ID,Number=number,Type=type,Description="description">
\end{verbatim}

Possible Types for FORMAT fields are: Integer, Float, Character, and String (this field is otherwise defined precisely as the INFO field).

\subsubsection{Alternative allele field format}
Symbolic alternate alleles should be described as follows:
\begin{verbatim}
##ALT=<ID=type,Description=description>
\end{verbatim}

\noindent \textbf{Structural Variants} \newline
In symbolic alternate alleles for imprecise structural variants,
the ID field indicates the type of structural variant, and can be a
colon-separated list of types and subtypes. ID values are case sensitive
strings and may not contain whitespace or angle brackets. The first level type
must be one of the following:
\begin{itemize}
  \item DEL Deletion relative to the reference
  \item INS Insertion of novel sequence relative to the reference
  \item DUP Region of elevated copy number relative to the reference
  \item INV Inversion of reference sequence
  \item CNV Copy number variable region (may be both deletion and duplication)
\end{itemize}

The CNV category should not be used when a more specific category can be applied. Reserved subtypes include:
\begin{itemize}
  \item DUP:TANDEM Tandem duplication
  \item DEL:ME Deletion of mobile element relative to the reference
  \item INS:ME Insertion of a mobile element relative to the reference
\end{itemize}

\bigskip

\noindent \textbf{IUPAC ambiguity codes} \newline
Symbolic alleles can be used also to represent genuinely ambiguous data in VCF, for example:
\begin{verbatim}
    ##ALT=<ID=R,Description="IUPAC code R = A/G">
    ##ALT=<ID=M,Description="IUPAC code M = A/C">
\end{verbatim}


\subsubsection{Assembly field format}
Breakpoint assemblies for structural variations may use an external file:
\begin{verbatim}
##assembly=url
\end{verbatim}

The URL field specifies the location of a fasta file containing breakpoint assemblies referenced in the VCF records for structural variants via the BKPTID INFO key.

\subsubsection{Contig field format}
\label{sec-contig-field}
It is highly recommended (and required for BCF) that the header includes tags
describing the contigs referred to in the VCF file. The structured \texttt{contig}
field must include the ID attribute and typically includes also
sequence length, MD5 checksum, URL tag to indicate where the sequence can be
found, etc. For example: 
\begin{verbatim}
##contig=<ID=ctg1,length=81195210,URL=ftp://somewhere.org/assembly.fa,...>
\end{verbatim}

\noindent
Valid contig names must follow the reference sequence names allowed by the SAM format
("{\tt [!-)+-\char60\char62-\char126][!-\char126]*}") excluding the characters "\texttt{\textless\textgreater[]:*}" to avoid clashes with
symbolic alleles and breakend notation.  The contig names must not use a reserved symbolic allele name.


\subsubsection{Sample field format}
It is possible to define sample to genome mappings as shown below:
{\scriptsize
\begin{verbatim}
##META=<ID=Assay,Type=String,Number=.,Values=[WholeGenome, Exome]>
##META=<ID=Disease,Type=String,Number=.,Values=[None, Cancer]>
##META=<ID=Ethnicity,Type=String,Number=.,Values=[AFR, CEU, ASN, MEX]>
##META=<ID=Tissue,Type=String,Number=.,Values=[Blood, Breast, Colon, Lung, ?]>
##SAMPLE=<ID=Sample1,Assay=WholeGenome,Ethnicity=AFR,Disease=None,Description="Patient germline genome from unaffected",DOI=url>
##SAMPLE=<ID=Sample2,Assay=Exome,Ethnicity=CEU,Disease=Cancer,Tissue=Breast,Description="European patient exome from breast cancer">
\end{verbatim}}

\subsubsection{Pedigree field format}
It is possible to record relationships between genomes using the following syntax:
\begin{verbatim}
##PEDIGREE=<ID=TumourSample,Original=GermlineID>
##PEDIGREE=<ID=SomaticNonTumour,Original=GermlineID>
##PEDIGREE=<ID=ChildID,Father=FatherID,Mother=MotherID>
##PEDIGREE=<ID=SampleID,Name_1=Ancestor_1,...,Name_N=Ancestor_N>
\end{verbatim}
\noindent or a link to a database:
\begin{verbatim}
##pedigreeDB=URL
\end{verbatim}

\noindent See \ref{PedigreeInDetail} for details.


\subsection{Header line syntax}
The header line names the 8 fixed, mandatory columns. These columns are as follows:

\begin{enumerate}
  \item \#CHROM
  \item POS
  \item ID
  \item REF
  \item ALT
  \item QUAL
  \item FILTER
  \item INFO
\end{enumerate}

If genotype data is present in the file, these are followed by a FORMAT column header, then an arbitrary number of sample IDs. The header line is tab-delimited
and there must be no tab characters at the end of the line.

\subsection{Data lines}
All data lines are tab-delimited
with no tab character at the end of the line. The last data line should end with a line separator. In all cases,
missing values are specified with a dot (`.').

\subsubsection{Fixed fields}
There are 8 fixed fields per record.  Fixed fields are:

\begin{enumerate}
  \item CHROM - chromosome: An identifier from the reference genome or an angle-bracketed ID String (``$<$ID$>$'') pointing to a contig in the assembly file (cf. the \#\#assembly line in the header). All entries for a specific CHROM should form a contiguous block within the VCF file. The colon symbol (:) must be absent from all chromosome names to avoid parsing errors when dealing with breakends. (String, no white-space permitted, Required).
  \item POS - position: The reference position, with the 1st base having position 1. Positions are sorted numerically, in increasing order, within each reference sequence CHROM.   It is permitted to have multiple records with the same POS. Telomeres are indicated by using positions 0 or N+1, where N is the length of the corresponding chromosome or contig.   (Integer, Required)
  \item ID - identifier: Semi-colon separated list of unique identifiers where available. If this is a dbSNP variant it is encouraged to use the rs number(s). No identifier should be present in more than one data record. If there is no identifier available, then the missing value should be used. (String, no white-space or semi-colons permitted, duplicate values not allowed.)
  \item REF - reference base(s): Each base must be one of A,C,G,T,N (case
  insensitive). Multiple bases are permitted. The value in the POS field refers
  to the position of the first base in the String. For simple insertions and
  deletions in which either the REF or one of the ALT alleles would otherwise
  be null/empty, the REF and ALT Strings must include the base before the event
  (which must be reflected in the POS field), unless the event occurs at
  position 1 on the contig in which case it must include the base after the
  event; this padding base is not required (although it is permitted) for e.g.
  complex substitutions or other events where all alleles have at least one
  base represented in their Strings.  If any of the ALT alleles is a symbolic
  allele (an angle-bracketed ID String ``$<$ID$>$'') then the padding base is
  required and POS denotes the coordinate of the base preceding the
  polymorphism. Tools processing VCF files are not required to preserve case in
  the allele Strings. (String, Required).

  If the reference sequence contains IUPAC ambiguity codes not
  allowed by this specification (such as R = A/G), the ambiguous reference base 
  must be reduced to a concrete base by using the one that is first alphabetically
  (thus R as a reference base is converted to A in VCF.)


  \item ALT - alternate base(s): Comma separated list of alternate non-reference alleles.  These alleles do not have to be called in any of the samples. Options are base Strings made up of the bases A,C,G,T,N,*, (case insensitive) or an angle-bracketed ID String (``$<$ID$>$'') or a breakend replacement string as described in the section on breakends. The `*' allele is reserved to indicate that the allele is missing due to a an overlapping deletion. If there are no alternative alleles, then the missing value should be used.  Tools processing VCF files are not required to preserve case in the allele String, except for IDs, which are case sensitive.  (String; no whitespace, commas, or angle-brackets are permitted in the ID String itself)
  \item QUAL - quality: Phred-scaled quality score for the assertion made in ALT. i.e. $-10log_{10}$ prob(call in ALT is wrong). If ALT is `.' (no variant) then this is $-10log_{10}$ prob(variant), and if ALT is not `.' this is $-10log_{10}$ prob(no variant). If unknown, the missing value should be specified. (Float)
  \item FILTER - filter status: PASS if this position has passed all filters, i.e. a call is made at this position. Otherwise, if the site has not passed all filters, a semicolon-separated list of codes for filters that fail. e.g. ``q10;s50'' might indicate that at this site the quality is below 10 and the number of samples with data is below 50\% of the total number of samples. `0' is reserved and should not be used as a filter String. If filters have not been applied, then this field should be set to the missing value. (String, no white-space or semi-colons permitted, duplicate values not allowed.)
  \item INFO - additional information: (String, no semi-colons or
  equals-signs permitted; commas are permitted only as delimiters for lists of
  values; characters with special meaning can be encoded using the percent encoding, see Section~\ref{character-encoding}; space characters are allowed)
  INFO fields are encoded as a semicolon-separated series of short keys
  with optional values in the format: $<$key$>$=$<$data$>$[,data]. 
  INFO keys must match the regular expression \texttt{\^{}[A-Za-z\_][0-9A-Za-z\_.]*\$}, duplicate fields are not allowed.
  Arbitrary keys are permitted, although the following sub-fields are reserved (albeit optional):
\begin{itemize}
  \item AA : ancestral allele
  \item AC : allele count in genotypes, for each ALT allele, in the same order as listed
  \item AD, ADF, ADR: read depths for each allele; total (AD), on the forward (ADF) and the reverse (ADR) strand (Integer, Number=R)
  \item AF : allele frequency for each ALT allele in the same order as listed: use this when estimated from primary data, not called genotypes
  \item AN : total number of alleles in called genotypes
  \item BQ : RMS base quality at this position
  \item CIGAR : cigar string describing how to align an alternate allele to the reference allele
  \item DB : dbSNP membership
  \item DP : combined depth across samples, e.g. DP=154
  \item END : end position of the variant described in this record (for use with symbolic alleles)
  \item H2 : membership in hapmap2
  \item H3 : membership in hapmap3
  \item MQ : RMS mapping quality, e.g. MQ=52
  \item MQ0 : Number of MAPQ == 0 reads covering this record
  \item NS : Number of samples with data
  \item SB : strand bias at this position
  \item SOMATIC : indicates that the record is a somatic mutation, for cancer genomics
  \item VALIDATED : validated by follow-up experiment
  \item 1000G : membership in 1000 Genomes
  \item $\ldots$ see Section~\ref{sv-info-keys} for a list of INFO keys reserved for structural variants.
\end{itemize}
\end{enumerate}
The exact format of each INFO sub-field should be specified in the meta-information (as described above).
Example for an INFO field: DP=154;MQ=52;H2. Keys without corresponding values are allowed in order to indicate group membership (e.g. H2 indicates the SNP is found in HapMap 2). It is not necessary to list all the properties that a site does NOT have, by e.g. H2=0. See below for additional reserved INFO sub-fields used to encode structural variants.
\subsubsection{Genotype fields}
If genotype information is present, then the same types of data must be present
for all samples. First a FORMAT field is given specifying the data types and
order (colon-separated FORMAT ids matching the regular expression \texttt{\^{}[A-Za-z\_][0-9A-Za-z\_.]*\$}, duplicate fields are not allowed). This is followed by one data block per
sample, with the colon-separated data corresponding to the types
specified in the format. The first sub-field must always be the genotype (GT)
if it is present.  There are no required sub-fields.

As with the INFO field, there are several common, reserved keywords that are standards across the community:

\begin{itemize}
\renewcommand{\labelitemii}{$\circ$}
  \item AD, ADF, ADR: per-sample read depths for each allele; total (AD), on the forward (ADF) and the reverse (ADR) strand (Integer, Number=R)
  \item DP : read depth at this position for this sample (Integer)
  \item EC : comma separated list of expected alternate allele counts for each alternate allele in the same order as listed in the ALT field (typically used in association analyses) (Integers)
  \item FT : sample genotype filter indicating if this genotype was ``called'' (similar in concept to the FILTER field). Again, use PASS to indicate that all filters have been passed, a semi-colon separated list of codes for filters that fail, or `.' to indicate that filters have not been applied. These values should be described in the meta-information in the same way as FILTERs (String, no white-space or semi-colons permitted)
  \item GQ : conditional genotype quality, encoded as a phred quality $-10log_{10}$ p(genotype call is wrong, conditioned on the site's being variant) (Integer)
  \item GP : genotype posterior probabilities in the range 0 to 1 using the same ordering as the GL field; one use can be to store imputed genotype probabilities (Float)
  \item GT : genotype, encoded as allele values separated by either of $/$ or $\mid$. The allele values are 0 for the reference allele (what is in the REF field), 1 for the first allele listed in ALT, 2 for the second allele list in ALT and so on. For diploid calls examples could be $0/1$, $1\mid0$, or $1/2$, etc. For haploid calls, e.g. on Y, male non-pseudoautosomal X, or mitochondrion, only one allele value should be given; a triploid call might look like $0/0/1$. If a call cannot be made for a sample at a given locus, `.' should be specified for each missing allele in the GT field (for example `$./.$' for a diploid genotype and `.' for haploid genotype). The meanings of the separators are as follows (see the PS field below for more details on incorporating phasing information into the genotypes):
	\begin{itemize}
	  \item $/$ : genotype unphased
	  \item $\mid$ : genotype phased
	\end{itemize}

  \item GL : genotype likelihoods comprised of comma separated floating point
  $log_{10}$-scaled likelihoods for all possible genotypes given the set of
  alleles defined in the REF and ALT fields. In presence of the GT field the
  same ploidy is expected; without GT field, diploidy is assumed. 

  \textsc{Genotype Ordering.} In general case of ploidy P and N alternate alleles (0 is the REF and 1..N
  the alternate alleles), the ordering of genotypes for the likelihoods can
  be expressed by the following pseudocode with as many nested loops as ploidy:\footnote{Note that we use inclusive \texttt{for} loop boundaries.}
  \begingroup
  \small
  \begin{lstlisting}
  for $a_P = 0\ldots N$
    for $a_{P-1} = 0\ldots a_P$
        $\ldots$
        for $a_1 = 0\ldots a_{2}$
            println $a_1 a_2  \ldots  a_P$
  \end{lstlisting}
  \endgroup

  Alternatively, the same can be achieved recursively with the following pseudocode:
  \begingroup
  \small
  \begin{lstlisting}
    Ordering($P$, $N$, suffix=""):
        for $a$ in $0\ldots N$
            if ($P == 1$) println str($a$) + suffix
            if ($P > 1$) Ordering($P$-1, $a$, str($a$) + suffix)
  \end{lstlisting}
  \endgroup

  Conversely, the index of the value corresponding to the genotype $k_1\le k_2\le\ldots\le k_P$ is
  \begingroup
  \small
  \begin{lstlisting}
    Index($k_1/k_2/\ldots/k_P$) = $\sum_{m=1}^{P} {k_m + m - 1 \choose m}$
  \end{lstlisting}
  \endgroup

  Examples:
    \begin{itemize}
    \item for $P$=2 and $N$=1, the ordering is 00,01,11
    \item for $P$=2 and $N$=2, the ordering is 00,01,11,02,12,22
    \item for $P$=3 and $N$=2, the ordering is 000, 001, 011, 111, 002, 012, 112, 022, 122, 222
    \item for $P$=1, the index of the genotype $a$ is $a$
    \item for $P$=2, the index of the genotype "$a/b$", where $a\le b$, is $b (b+1)/2 + a$
    \item for $P$=2 and arbitrary $N$, the ordering can be easily derived from a triangular matrix
            \newline
            \hbox{\hskip5em\footnotesize
            \begin{tabular}{l|llll}
               $b\setminus a$ & 0 & 1 & 2 & 3 \\ \hline \\[-0.5em]
               0   & 0 &   &   &   \\
               1   & 1 & 2 &   &   \\
               2   & 3 & 4 & 5 &   \\
               3   & 6 & 7 & 8 & 9 
            \end{tabular}
            }
    \end{itemize}

  \item HQ : haplotype qualities, two comma separated phred qualities (Integers)
  \item MQ : RMS mapping quality, similar to the version in the INFO field. (Integer)
  \item PL : the phred-scaled genotype likelihoods rounded to the closest integer (and otherwise defined precisely as the GL field) (Integers)
  \item PQ : phasing quality, the phred-scaled probability that alleles are ordered incorrectly in a heterozygote (against all other members in the phase set).  We note that we have not yet included the specific measure for precisely defining ``phasing quality''; our intention for now is simply to reserve the PQ tag for future use as a measure of phasing quality. (Integer)
  \item PS : phase set.  A phase set is defined as a set of phased genotypes to which this genotype belongs.  Phased genotypes for an individual that are on the same chromosome and have the same PS value are in the same phased set.  A phase set specifies multi-marker haplotypes for the phased genotypes in the set.  All phased genotypes that do not contain a PS subfield are assumed to belong to the same phased set.  If the genotype in the GT field is unphased, the corresponding PS field is ignored.  The recommended convention is to use the position of the first variant in the set as the PS identifier (although this is not required). (Non-negative 32-bit Integer)
  \item $\ldots$ see Section~\ref{sv-format-keys} for a list of genotype keys reserved for structural variants.
\end{itemize}


If any of the fields is missing, it is replaced with the missing value. For example if the FORMAT is GT:GQ:DP:HQ then $0\mid0:.:23:23,34$ indicates that GQ is missing. Trailing fields can be dropped (with the exception of the GT field, which should always be present if specified in the FORMAT field).

See below for additional genotype fields used to encode structural variants. Additional Genotype fields can be defined in the meta-information. However, software support for such fields is not guaranteed.