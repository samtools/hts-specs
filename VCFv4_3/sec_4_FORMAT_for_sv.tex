\section{FORMAT keys used for structural variants}
\label{sv-format-keys}
\footnotesize
\begin{verbatim}
##FORMAT=<ID=CN,Number=1,Type=Integer,Description="Copy number genotype for imprecise events">
##FORMAT=<ID=CNQ,Number=1,Type=Float,Description="Copy number genotype quality for imprecise events">
##FORMAT=<ID=CNL,Number=G,Type=Float,Description="Copy number genotype likelihood for imprecise events">
##FORMAT=<ID=CNP,Number=G,Type=Float,Description="Copy number posterior probabilities">
##FORMAT=<ID=NQ,Number=1,Type=Integer,Description="Phred style probability score that the variant is novel">
##FORMAT=<ID=HAP,Number=1,Type=Integer,Description="Unique haplotype identifier">
##FORMAT=<ID=AHAP,Number=1,Type=Integer,Description="Unique identifier of ancestral haplotype">
\end{verbatim}
\normalsize
These keys are analogous to GT/GQ/GL/GP and are provided for genotyping
imprecise events by copy number (either because there is an unknown number of
alternate alleles or because the haplotypes cannot be determined). CN specifies
the integer copy number of the variant in this sample. CNQ is encoded as a
phred quality $-10log_{10}$ p(copy number genotype call is wrong). CNL
specifies a list of $log_{10}$ likelihoods for each potential copy number,
starting from zero. CNP is 0 to 1-scaled copy number posterior
probabilities (and otherwise defined precisely as the CNL field), intended to
store imputed genotype probabilities. When possible, GT/GQ/GL/GP should be used
instead of (or in addition to) these keys.