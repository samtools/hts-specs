\section{List of changes}

\subsection{Changes between VCFv4.2 and VCFv4.3}

\begin{itemize}
\item VCF compliant implementations must support both LF and CR+LF newline conventions
\item INFO and FORMAT tag names must match the regular expression \texttt{\^{}[A-Za-z\_][0-9A-Za-z\_.]*\$}
\item Spaces are allowed in INFO field values
\item Characters with special meaning (such as ';' in INFO, ':' in FORMAT, and '\%' in both) can be encoded using the percent encoding (see Section~\ref{character-encoding})
\item The character encoding of VCF files is UTF-8.
\item The SAMPLE field can contain optional DOI URL for the source data file
\item Introduced \#\#META header lines for defining phenotype metadata
\item New reserved tag "CNP" analogous to "GP" was added. Both CNP and GP use 0 to 1 encoding, which is a change from previous phred-scaled GP.
\item In order for VCF and BCF to have the same expressive power, we state explicitly that Integers and Floats are 32-bit numbers. Integers are signed.
\item We state explicitly that zero length strings are not allowed, this includes the CHROM and ID column, INFO IDs, FILTER IDs and FORMAT IDs. Meta-information lines can be in any order, with the exception of \#\#fileformat which must come first. 
\item All header  lines of the form \#\#key=$<$ID=xxx,...$>$ must have an ID value
that is unique for a given value of "key". All header lines whose value starts
with "$<$" must have an ID field. Therefore, also \#\#PEDIGREE newly requires a unique ID.
\item We state explicitly that duplicate IDs, FILTER, INFO or FORMAT keys are not valid.
\item A section about gVCF was added, introduced the $<$*$>$ symbolic allele.
\item A section about tag naming conventions was added.
\item New reserved AD, ADF, and ADR INFO and FORMAT fields added.
\item Removed unused and ill-defined GLE FORMAT tag.
\item Chromosome names cannot use reserved symbolic alleles and contain characters used by breakpoints (Section~\ref{sec-contig-field}).
\item IUPAC ambiguity codes should be converted to a concrete base.
\item Symbolic ALTs for IUPAC codes.
\end{itemize}

\subsection{Changes between BCFv2.1 and BCFv2.2}
\begin{itemize}
\item BCF header lines can include optional IDX field
\item We introduce end-of-vector byte and reserve 8 values for future use
\item Clarified that except the end-of-vector byte, no other negative values are allowed in the GT array 
\item String vectors in BCF do not need to start with comma, as the number of values is indicated already in the definition of the tag in the header.
\item The implicit filter PASS was described inconsistently throughout BCFv2.1: It is encoded as the first entry in the dictionary, not the last.
\end{itemize}